%%%%%%%%%%%%%%%%%%%%%%%%%%%%%%%%%%%%%%%%%%%%%%%%%%%%%%%%%%%%%%%%%%%%%%%%%%%
% This is a sample header for a sample dissertation. Fill in the name,
% and the other information. LaTeX will work out the table of
% content, the list of figures and of tables for you.
%%%%%%%%%%%%%%%%%%%%%%%%%%%%%%%%%%%%%%%%%%%%%%%%%%%%%%%%%%%%%%%%%%%%%%%%%%%


\newpage
\thispagestyle{empty}

% ******* Title page *******
% **************************

\vspace*{2cm}
\begin{center}
{\Large\bf Automatic Classification Of Optical Coherence Tomography Images\\} \vspace{2cm} {\large
Khaled Abdulhameed Manea Alsaih\\
\vspace{2cm}
Department of LE2I - Laboratoire Electronique, Informatique et Image \\
Universit\'e Bourgogne Franche-Comt\'e \\
University of Technology Malaysia }
\end{center}

\vspace{7cm}
\begin{center}
{\large A Thesis Submitted for the Degree of \\MSc
in Vision and Robotics (MSCV) \\\vspace{0.3cm} $\cdot$ 2016
$\cdot$}
\end{center}
\singlespacing


%ABSTRACT
\begin{abstract}
We study in this thesis new methods in order to analyse Optical Coherence Tomography (OCT) volumes in patients that suffer from Diabetes.
Early detection of Diabetic Macular Edema (DME) and Age-related Macular Degeneration (AMD) would help patients to get special treatments.
Thus, developing a mechanism or algorithm to automatically detecting the retinopathy diabetes is a must to save humans lives and maintain their status healthy. 

Basically, the algorithm for automated detection of the retina diseases is going through several steps depends on the task wanted to detect.
This project classifies volumes to diseased or normal in the first task, and classifies patches inside the images to cyst or background in the second task. 
First task is starting by preprocessing, which involves de-noising then flattening then cropping each B-scan of OCT images. After that, Multi-scale histograms of oriented gradient descriptors (HOG) and Local Binary Pattern (LBP) are extracted separately.
In other tests, HOG and LBP are combined to create a set of different feature vectors.
We also made some tests of the features vectors when projected into a lower-dimensional space through Principal Component Analysis (PCA).
Eventually, a Bag of Words (BoW) is also performed prior to feed the data to different classifiers like Linear SVM, kernel RBF SVM and Random Forest (RF).
Experimental results show a promising performance in terms of sensitivity (SE) and specificity (SP) of 87.5\% and 87.5\%, respectively, on a challenging dataset.
Another approach for classifying on OCT images is done by extracting Maximally Stable Extremal Regions (MSER) and label the regions extracted using the STAPLE ground truth to fit the labelled data to the autoencoder for learning and extracting the features.
After that, the output data of autoencoder are assigned to the softmax layer classifier to classify the cyst patches and the background patches.
Experimental results also show a promising performance in terms of sensitivity (SE) and specificity (SP) of 95.0\% and 79.0\%, respectively, on another challenging dataset provided by OPTIMA.
%$\lambda = \phi$ \ref{eq:eq1}
%\cite{}


\vspace*{5cm}



%\begin{center}
%\begin{quote}
%\it Research is what I'm doing when I don't know what I'm
%doing.\,\ldots
%\end{quote}
%\end{center}
%\hfill{\small Werner von Braun}

\end{abstract}

\doublespacing

%\pagestyle{empty}
\pagenumbering{roman}
\setcounter{page}{1} \pagestyle{plain}


\tableofcontents

\listoffigures
\listoftables \clearpage
\addcontentsline{toc}{chapter}{List of Abbreviations}
\printacronyms[name = Abbreviations]

\chapter*{Acknowledgments}
\addcontentsline{toc}{chapter}
         {\protect\numberline{Acknowledgments\hspace{-96pt}}}

First of all, I would like to thank my amazing supervisor Prof. Fabrice Meriaudeau for his consistent support and clear directions.
He never felt annoyed from my questions along with Dr.Desire Sidibe as well.
I would also love to thank my colleagues Guillaume Lema\^{i}tre and Mojdeh Rastgoo for their practical help through the duration of my master project.

I would like to send this thesis work and efforts to My country YEMEN and then to my father Prof. Abdulhameed Alsaih and My Mom for their permanent support through my life.
Also I would like to thank my Sister Taif,my brothers Tariq, Mojahed,Dr.Mohammed,Dr.Sakher, Alhalali Ali Almansour and Sabahi.

In addition, I would like to thank my friends that they Always support me starting by Hossam Hassan, Awss,Hamed alboori, Hashem Alsaih, Hamodi, Yousef, Zezo,Sufian, Seper, Paola, Jose, Dennis, Sandeep, Cansen, Raed, Kiril, and others.

As usual, I would like to thank my Seniors Dr.Fares, Dr.Taha and Dr.Abdulaziz for their motivation and help for any inquiries.   

Finally, thanks to University of Burgundy and University of Technology Malaysia for the great chance to be part of this course which taught me a lot and looking forward to apply it with the sense of serving humanity in a good manner.

\pagestyle{fancy}


